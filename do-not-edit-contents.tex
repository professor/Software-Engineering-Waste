% Sample apostrophy's to remove team's 



\begin{abstract}
\textit{Context:} Since software development is a complex socio-technical activity that involves coordinating different disciplines and skill sets, it provides ample opportunities for waste to emerge. Waste is any activity that when is removed or simplified does not affect the value to any stakeholder.produces no value for the customer or user.

\textit{Objective:} The purpose of this paper is to identify and describe different types of waste in software development.

\textit{Method:} Following Constructivist Grounded Theory, we conducted a \durationOfResearchStudy{} participant-observation study of \numberOfObservedProjects{} software development projects at Pivotal, a software development consultancy. We also interviewed \numberOfInterviews{} software engineers, interaction designers, and product managers, and analyzed one year of retrospection topics. We iterated between analysis and theoretical sampling until achieving theoretical saturation.

\textit{Results:} This paper introduces the first empirical waste taxonomy. It identifies \numberOfWastes{} wastes and explores their causes, underlying tensions, and overall relationship to the waste taxonomy found in Lean Software Development.

\textit{Limitations:} Grounded Theory does not support statistical generalization. While the proposed taxonomy appears widely applicable, organizations with different software development cultures may experience different waste types.

\textit{Conclusion:} Software development projects manifest \numberOfWastes{} types of waste: building the wrong feature or product, mismanaging the backlog, rework, unnecessarily complex solutions, extraneous cognitive load, psychological distress, waiting, knowledge loss, and ineffective communication.
 

%\sout{The taxonomy is the foundation for Value Stream Mapping, a process of examining value creation in software development process for the purpose of waste removal. Since waste removal is core to the Lean movement, starting with an ill formulated waste taxonomy suggests a possible rethinking of Lean Software Development claims.}



\end{abstract}

\begin{IEEEkeywords}
Software development waste,
Extreme Programming,
Lean Software Development
\end{IEEEkeywords}

% For peer review papers, you can put extra information on the cover
% page as needed:
% \ifCLASSOPTIONpeerreview
% \begin{center} \bfseries EDICS Category: 3-BBND \end{center}
% \fi
%
% For peerreview papers, this IEEEtran command inserts a page break and
% creates the second title. It will be ignored for other modes.
\IEEEpeerreviewmaketitle

\section{Introduction}
\participantQuote{The engineers are depressed. The project grinds them down\ldots It is hard to know which problem to tackle first. There is coupling everywhere\ldots Each layer of the system has unnecessary complexity\ldots The depth of knowledge about the system is meager\ldots There is a lot of waiting\ldots Building the java code takes ten minutes. Starting the server takes seven minutes. Running the javascript tests take two minutes. Running the integration tests take 47 minutes. Continuous integration takes \textit{forever} to run all the tests and get the code onto the acceptance environment. \newline \indent There is waste everywhere} \textemdash Software Engineer on Project Septem.

Software development is a complex socio-technical activity that involves coordinating different disciplines and skill sets. Identifying user needs, crafting features for those needs, identifying and prioritizing value, implementing features, releasing and supporting products provide ample opportunity for waste to creep in. 

Here, \quotes{waste} refers to \quotes{any activity that consumes resources but creates no value} for customers \cite{WomackLeanThinking}. Reducing waste, by definition, improves efficiency and productivity. Waste is like friction in the development process.

However, reducing waste is difficult not least because \textit{identifying} waste is difficult.  Numerous cognitive phenomena including status quo bias \cite{JostDecadeOfSystemJustification} hinder practitioners' propensity and ability to notice waste in existing practices. Identifying the types of waste that often occur in software projects  may, therefore, facilitate reducing waste. Identifying and eliminating waste is a key principle of lean manufacturing. 

The Toyota Production System \cite{OhnoToyotaProductionSystem, ShingoToyotaProductionSystem} transformed manufacturing from batch-and-queue to just-in-time. The similarities between batch-and-queue and waterfall software development, as well as just-in-time and iterative software development, inspired several software development methods \cite{PoppendieckLeanSoftwareDevelopment, AndersonKanban}. These methods adapt, in a top-down fashion, lean principles for software environments. 

However, manufacturing differs from software development in significant ways. Manufacturing produces physical products; software is intangible. While the 1000th car costs about as much to make as the 999th car, the marginal cost of the 1000th copy of a mobile app is near zero. While most factories build batches of near-identical goods, much software remains unique. 

Given the obvious differences between developing software and manufacturing physical products, software development may entail waste types never envisioned in lean manufacturing. Even the most careful adaptation of lean principles for software may not have identified such waste types. We therefore conducted an in-depth, longitudinal investigation of a successful software company to address the following research question: 

\textbf{Research Question: \quotes{What types of waste are observable  in software development?}}

Next, we summarize the history of lean and review related work. Section \ref{ResearchMethod} describes the research method. Section \ref{SEWaste} presents the emergent waste taxonomy. Section \ref{LeanSoftwareDevelopmentComparison} compares this model with the waste list from Lean Software Development. Sections \ref{ResultsEvaluation} and \ref{Conclusion} evaluate the results, describe limitations, and conclude the paper.

\section{A Brief History of Lean}
\label{HistoryOfLean}
Lean Thinking is a concept proposed by Womack \cite{WomackLeanThinking} following his analysis of The Toyota Production System. The Toyota Production System prioritizes waste removal by creating a culture that pursues waste identification and elimination in the entire production of a vehicle \cite{OhnoToyotaProductionSystem, ShingoToyotaProductionSystem}. In 1945, Toyota optimized for the production rate of each system, keeping like machines near each other. Ohno rearranged equipment so that the output of one machine fed into the next machine, slowed machines down to have the same cadence, and only produced material when it was needed. After optimizing Toyota's factories, Toyota then trained their suppliers so that the entire production of a vehicle was just-in-time, transforming from mass production to lean production. The resulting \quotes{pull} system was easy to reconfigure, minimized inventory, and supported short production runs.  

Lean Thinking describes a process of identifying and removing waste in a value stream \cite{WomackLeanThinking}. The process discerns three types of activities: activities that clearly create value; activities that create no value for the customer but are currently necessary to manufacture the product; and activities that create no value for the customer, are unnecessary and therefore should be removed immediately; i.e., waste. 

%Based on analysis of the Toyota Production System, Lean Thinking \cite{WomackLeanThinking} describes a process of identifying and removing waste using five principles:
%\begin{enumerate}
%\item Specify value: define value from the customer's perspective
%\item Identify the value stream: examine all actions required to bring raw materials to final product for the customer and eliminate any obvious unnecessary steps
%\item Flow: re-engineer from batch-and-queue to just-in-time or continuous flow 
%\item Pull: create products only in response to a customer order
%\item Perfection: continue with a continuous process of waste identification and elimination
%\end{enumerate}

The Toyota Production System characterized seven types of manufacturing waste \cite{ShingoToyotaProductionSystem} shown in Table \ref{ManufacturingWaste}. Later, Womack and Liker each added a waste type: value and non-utilized talent \cite{WomackLeanThinking, LikerToyotaWay}.

\begin{table}[t]
\renewcommand{\arraystretch}{1.5}
\centering
\caption{Toyota Production System Definition of Manufacturing Waste}
\label{ManufacturingWaste}
% \begin{tabular}{|LEFT{1.15in}|p{2.00in}|}
\begin{tabular}{| >{\raggedright}m{1.15in}|p{2.00in}|}
% \begin{tabular}{|p{2in}|p{4in}|} % dissertation size
\hline
Waste Type                & Description                                                                                                                                                  \\ \hline
Inventory                 & The cost of storing materials until they are needed. The material might never used.                                                                   \\ \hline
Extra Processing          & The cost of processing that is not needed by a downstream step in the manufacturing process. (Sometimes an inefficiency from not seeing the entire process.) \\ \hline
Overproduction            & The cost of producing more quantity of components than necessary for the present.                                                                            \\ \hline
Transportation (of goods) & The cost of unnecessarily moving materials from one place to another place.                                                                                  \\ \hline
Waiting                   & The cost of waiting for a previous upstream step to finish.                                                                                                       \\ \hline
Motion (of people)        & The cost of unnecessary picking up and putting things down.                                                                                                  \\ \hline
Defects                   & The cost of rework from quality defects.                                                                                                                     \\ \hline
Value (added by \cite{WomackLeanThinking})                     & The cost of producing goods and services that do not meet the needs of the customer.                                                                         \\ \hline
Non-utilized Talent (added by \cite{ LikerToyotaWay})      & The cost of unused employee creativity and talent.                                                                                                           \\ \hline
\end{tabular}
\end{table}


Mary and Tom Poppendieck created Lean Software Development \cite{PoppendieckLeanSoftwareDevelopment}, by adapting Lean Thinking and the Toyota Production System from manufacturing to software development. Their comparison of manufacturing waste with software waste is presented in Table \ref{ManufacturingVersusLeanSoftwareWaste}.

\begin{table}[t]
\renewcommand{\arraystretch}{1.5}
\centering
\caption{Comparison of Manufacturing Waste with Lean Software Development Waste}
\label{ManufacturingVersusLeanSoftwareWaste}
\begin{tabular}{|p{1.57in}|p{1.57in}|}
% \begin{tabular}{|l|l|} %dissertation size
\hline
Toyota Production System's Manufacturing Wastes & Lean Software Development Wastes \cite{PoppendieckConceptToCash} \\ \hline
Inventory                                       & Partially Done Work                       \\ \hline
Extra Processing                                & Relearning                                \\ \hline
Overproduction                                  & Extra Features                            \\ \hline
Transportation (of goods)                       & Handoffs                                  \\ \hline
Waiting                                         & Delays                                    \\ \hline
Motion (of people)                              & Task Switching                            \\ \hline
Defects                                         & Defects                                   \\ \hline
Value (added by cite{WomackLeanThinking})                 & N/A                                       \\ \hline
Non-utilized Talent (added by \cite{ LikerToyotaWay})     & N/A                                       \\ \hline
\end{tabular}
\end{table}

Adapting a taxonomy from a reference discipline (e.g. manufacturing) for a target discipline (e.g. software engineering) manifests at least four threats to validity: 
The target domain may include concepts (wastes) not found in the source domain 
The source domain may include concepts not found in the target domain
Concepts from the source domain may bias our perception of superficially similar but fundamentally different concepts in the target domain
The organization of concepts in the source domain may not fit the target domain (e.g., two or more manufacturing wastes might map into a single software engineering waste or vice versa) 
It is therefore incumbent upon researchers to empirically evaluate concepts, taxonomies and theories adapted from reference disciplines. We are not aware of any direct empirical validation of the Lean Software Development waste taxonomy, which motivates the current study. 

That said, several studies have used the Lean Software Development waste model. For example, Power and Conboy combine it with literature in manufacturing, lean production, product development, construction, and healthcare. They shift from using wastes of inefficiencies to impediments to flow. \cite{PowerImpediments}

%Petersen and Wohlin, consequently examine the flow of features by creating cumulative flow diagrams through different phases of software development observed at Ericsson AB in Sweden and India. (The phases are detailing features, implementing and unit testing features, isolation testing, system testing, and ready for release.) They defined several metrics to identify bottlenecks, variance in hand-overs, and cost types. For cost savings analysis, they define waste as any feature that has work done on it (e.g. \quotes{describing the feature}) but is never released to a customer \cite{Petersen2011}.

Several studies applied Value Stream Mapping to software development. Value Stream Mapping popularized by Womack systematically examines each stage for waste. Interestingly, these studies only found \textit{waiting} waste generated in a batch-and-queue system \cite{Ali2016, Khurum2014, Mujtaba2010}. One study identified the wastes of motion and extra processing from interviews, not the current state map \cite{Mujtaba2010}.These studies typically reduced waste by switching the organization from waterfall to iterative software development or reducing the batch size in iterative software development \cite{Ali2016, Khurum2014, Mujtaba2010}.

%Khurum said, \quotes{the researchers found it is more suitable to start focusing on improvement potential based on long waiting or lead time.} During the waste identification step of the workshop with their research participants, they ask attendees \quotes{in which phase do we see the majority of waiting?} \cite{Khurum2014} The Pygmalion effect (self-fulfilling prophecy) may explain why the researchers only found \textit{waiting} waste in Value Stream Mapping analysis.

%Ali et al applied information flow modeling to Value Stream Mapping which revealed \textit{waiting} waste from the passing of big batches from group to group and missing prioritization \cite{Ali2016}.

\section{Research Method}
\label{ResearchMethod}

We used Constructivist Grounded Theory \cite{Charmaz}, which involves iteratively collecting and analyzing data to generate and refine an emergent theory. We began by asking, \quotes{What is happening at Pivotal Labs when it comes to software development?}. This led to the \textit{Theory of Sustainable Software Development} \cite{SedanoSustainableSoftware} and two further core categories: \textit{Team Code Ownership} \cite{SedanoTeamCodeOwnership} and the topic of this paper: \textit{Removing Waste}.

Initially, the two primary data sources were participant observation field notes and interviews with Pivotal software engineers, interaction designers, and product managers. Interviews were recorded, transcribed, coded, and analyzed using constant comparison. The data advanced from initial codes to focused codes, focused codes to core categories.

When \textit{Removing Waste} emerged as a core category, we incorporated data from retrospection meetings, performed additional interviews, and continued participant observation to refine the category.  We constantly compared emerging findings to data from these three data sources, until reaching saturation, as described below.

\subsection{Research Context: Pivotal Labs}
We selected Pivotal Labs as the research context because it is a successful software engineering organization, interested in using and evolving extreme programming, and open to research collaboration. 

Pivotal Labs is a division of Pivotal\textemdash a large American software company (with 17 offices around the world). Pivotal Labs provides teams of agile developers, product managers, and interaction designers to other firms. Its mission is not only to deliver highly-crafted software products but also to help transform clients' engineering cultures. To change the client's development process, Pivotal combines the client's software engineers with Pivotal's engineers at a Pivotal office where they can experience Extreme Programming \cite{BeckExtremeProgramming2004} in an environment conducive to agile development. 

Typical teams include six software engineers, one interaction designer, and a product manager. The largest project in the history of the Palo Alto office had 28 developers while the smallest had two. Larger projects are organized into smaller coordinating teams with one product manager per team and one or two interaction designers per team.

Interaction designers identify user needs predominately through user interviews; create and validate user experience with mockups; determine the visual design of a product; and support engineering during implementation. Product managers are responsible for identifying and prioritizing features, converting features into stories, prioritizing stories in a backlog, and communicating the stories to the engineers. Software engineers implement the solution. 

Pivotal Labs has practiced Extreme Programming \cite{BeckExtremeProgramming2004} since the late 1990's. While each team autonomously decides what is best for each project, the company culture strongly suggests following all of the core practices of Extreme Programming, including pair programming, test-driven development, weekly retrospectives, daily stand-ups, a prioritized backlog, and team code ownership. 

We only observed teams at Pivotal Labs. Other teams, especially teams in other divisions, might have a different culture and follow different software practices.

\subsection{Data Collection}
This paper analyses data from three sources: 1) participant observation of \numberOfObservedProjects{} projects over \durationOfResearchStudyPlural{}, 2) interviews with Pivotal employees and 3) topics discussed in 91 retrospection meetings. 
\subsubsection{Participant Observation}
The first author collected field notes while working as an engineer on \numberOfObservedProjects{} projects. These notes describe individual and collective actions, capture what participants found interesting or problematic, and include anecdotes and observations.

Projects are de-identified to preserve client confidentiality:

\begin{itemize}
\item Project Unum (two product managers, four developers) was a greenfield project providing a web front end for installation, configuring, and using a multi-node cluster with big data tools. 
\item Project Duo (two interaction designers, two product managers, six developers) added features to a print-on-demand e-commerce platform. 
\item Project Tes (one interaction designer, one product manager, six developers) added features to management software for internet service providers.
\item Project Quattuor (two interaction designers, three product managers, 28 developers) developed two mobile applications and a backend system for controlling expensive equipment.
\item Project Kvin (one interaction designer, one product manager, six developers) was a greenfield project for a healthcare startup. 
\item Project Ses (two interaction designers, one product manager, ten developers) was adding features and removing technical debt to an existing internet e-commerce website.
\item Project Septem (two interaction designers, three product managers, twelve developers) was adding features and removing technical debt to an existing virtual machine management software.
\item Project Octo (one product manager, four developers) added features for workload management of a multi-node database.
\end{itemize}
\subsubsection{Interviews}
The first author interviewed \numberOfInterviews{} interaction designers, product managers, and software engineers who had experience with Pivotal's software development process from five different Pivotal offices. Participants were not paid for their time.

We relied on \quotes{intensive interviews,} which are \quotes{open-ended yet directed, shaped yet emergent, and paced yet unrestricted} \cite{Charmaz}. Open-ended questions were used to enter into the participant's personal perspective within the context of the research question. The interviewer attempts to abandon assumptions to better understand and explore the interviewee's perspective. 

The initial interviews began with the question, \quotes{Please draw on this sheet of paper your view of Pivotal's software development process.} The interviewer intentionally avoided forcing initial topics. While exploring new emergent core categories, whenever possible, we initiated subsequent interviews with open-ended questions. The first author transcribed each interview with timecode stamps for each segment. These interviews were spread across the duration of the research study. 
\subsubsection{Retrospection Topics}
When \textit{removing waste} emerged as a core category, we began collecting data from retrospection meetings. A retrospection meeting (or retro) is a meeting to pause, reflect, and discuss the work done during the week, i.e., a safe place where any team member can discuss any issue \cite{DerbyAgileRetrospectives}. Retros are typically scheduled every Friday afternoon. The entire team and important stakeholders attend these meetings. 

The observed Pivotal teams mostly use an emotion-based retro format where \quotes{happy,} \quotes{neutral,} and \quotes{sad} faces are written on the top of a whiteboard. The happy-face column represents items that are working well and should be continued or expanded. The meh-face column represents items that the team needs to \quotes{keep an eye on.} The sad-face column represents problems that the team should try to fix. Any team member can add any topic to any column. After a few minutes, the team dot-votes on the topics to discuss \cite{DerbyAgileRetrospectives}. The team uses the remainder of the sixty-minute meeting to discuss topics. Sometimes discussing a topic is sufficient to affect change, other times the team creates action items. 

We collected data from 91 retrospection meetings over 59 weeks from Projects Quattuor, Kvin, and Ses. (There are more meetings than weeks since each of Project Quattuor's three teams held its own retro each week.)

For co-located teams, the first author took a picture of the whiteboard at the end of the retro and later transcribed the topics into a master spreadsheet. For distributed teams, we copied data from the on-line spreadsheets the team used in place of a whiteboard. Attendees often wrote a short phrase as a proxy for a larger idea (e.g.,  \quotes{Scope} represents \quotes{Too much scope is causing the team stress}). When the provided topic was too vague, we solicited a more detailed description from an engineer present in the meeting. This produced 663 total items for analysis. 
\subsection{Data Analysis}
We began by iteratively collecting and analyzing field notes and interviews. We used line-by-line coding \cite{Charmaz} to identify nuanced interactions in the data and avoid jumping to conclusions. We reviewed the initial codes while reading the transcripts and listening to the audio recordings. We discussed the coding during weekly research collaboration meetings. To avoid missing insights from these discussions \cite{GlaserTheoreticalSensitivity}, we recorded and transcribed them into Grounded Theory memos. As data was collected and coded, we stored initial codes in a spreadsheet and we used constant comparison to generate focused codes.

We routinely compared new codes to existing codes to refine codes and eventually generate categories. We periodically audited each category for cohesion by comparing its codes. When this became complex, we printed codes on index cards to facilitate reorganization. We wrote memos to capture the analysis of codes, examinations of theoretical plausibility, and insights.

When \textit{removing waste} appeared as a core category, we analyzed data from retrospectives to investigate (theoretical sampling). After removing irrelevant topics (e.g. complaints about the weather), we printed each retro item onto an index card with its original retro topic, enhanced description, ID, and team name (see Figure \ref{exampleRetroTopicl}).

Two researchers with first-hand experience of the projects coded the retro topics and merged duplicate topics. We iteratively reorganized categories, keeping similar items together and dissimilar items apart. Figure \ref{ChainOfEvidence} gives an example classification for the \textit{psychological distress} waste. The figure shows the waste category, its cause categories and properties, and examples of observed retrospective topics illustrating the waste. The full chain of evidence is available \cite{SedanoDissertation}. 

We often stopped to record new insights. When the categories began to stabilize, we compared each category against the other categories looking for relationships. Once we felt that the categories were stable, we performed a final review of each category to verify that the cards belonged to it. We continued theoretical sampling for removing waste in additional interviews and participant observations until no further waste-related categories were evident, i.e. theoretical saturation. 

\begin{figure}[t]
\centering
\fbox{\includegraphics[width=3in]{waste_images/retro_index_card.png}}
\caption{Example Retro Topic Index Card}
\label{exampleRetroTopicl}
\end{figure}














\begin{table}[ht]
\centering
\captionof{figure}{Waste Organization Example (Psychological Distress)}
\label{ChainOfEvidence}
\begin{tabular}{|lll|}
\hline
\multicolumn{3}{|l|}{}  \\
\multicolumn{3}{|l|}{Waste category: Psychological Distress}  \\
     & \multicolumn{2}{l|}{Cause property: Low team morale} \\
     &      & Retro Topic: Frustrated clients / Pivotal developers       \\
     &      & Retro Topic: Negative attitudes                \\
     &      & Retro Topic: Apathy                            \\
     &      & Retro Topic: Unacknowledged by management      \\
     &      & Retro Topic: Messy code                        \\
     &      & Retro Topic: Pairing fatigue                   \\
     &      & Retro Topic: Poor lighting, lack of windows    \\
     & \multicolumn{2}{l|}{Cause property: Rush mode} \\
     &      & Retro Topic: Fixed features with a fixed timeline \\
     &      & Retro Topic: Aggressive timelines \\
     &      & Retro Topic: Scope creep \\
     &      & Retro Topic: Repeatedly saying \quotes{This is due today} \\
     &      & Retro Topic: Long days \\
     &      & Retro Topic: Overtime \\
     & \multicolumn{2}{l|}{Cause property: Lack of empathy} \\
     &      & Retro Topic: Not listening \\
     &      & Retro Topic: Criticizing in public \\
     &      & Retro Topic: Difficult pairings \\
     &      & Retro Topic: Interpersonal conflict \\
     &      & Retro Topic: Kicking product out of the team space \\
\hline
\end{tabular}
\end{table}



\section{Results: Types of Waste in Software Engineering}
\label{SEWaste}

\begin{table*}[t]
\renewcommand{\arraystretch}{1.3}
\centering
\caption{Types of Software Development Waste}
\label{Waste}
% \begin{tabular}{|p{1.2in}|p{1.9in}|p{3.5in}|} 
\begin{tabular}{|p{1.5in}|p{1.9in}|p{3.2in}|}
% \begin{tabular}{|p{1.7in}|p{1.9in}|p{3in}|} %original draft size
% \begin{tabular}{|p{1.5in}|p{1.6in}|p{2.8in}|} %dissertation size
\hline
Waste  & Description & Observed Causes                                                                                                                                                                                                                                                                                                                                                                                                                     \\ \hline
Building the wrong feature or product &  The cost of building a feature or product that does not address user or business needs. & 
User Desiderata (not doing user research, validation, or testing; ignoring user feedback; working on low user value features) \newline Business Desiderata (not involving a business stakeholder; slow stakeholder feedback; unclear product priorities)                                                                                                                                                                                  \\ \hline
Mismanaging the backlog     & The cost of duplicating work, expediting lower value user features, or delaying necessary bug fixes.  & 
Backlog inversion \newline Working on too many features simultaneously \newline Duplicated work \newline Not enough ready stories  \newline Imbalance of feature work and bug fixing \newline Delaying testing or critical bug fixing \newline Capricious thrashing                                                                                                                                                                                                                                                                                                                                    \\ \hline
Rework                                & The cost of altering delivered work that should have been done correctly but was not & Rejected Stories (product manager rejects story implementation) \newline No clear definition of done (ambiguous stories; second guessing design mocks) \newline Defects and Bugs (poor testing strategy; no root-cause analysis on bugs)                                                                                                                                                                   \\ \hline
Unnecessarily complex solutions  & The cost of creating a more complicated solution than necessary,  a missed opportunity to simplify features, user interface, or code.      & 
Unnecessary feature complexity from the user's perspective; unnecessary technical complexity (duplicating code, lack of interaction design reuse, overly complex technical design created up-front)
\\ \hline
Extraneous cognitive load &  The costs of unneeded expenditure of mental energy  &  
Complex or large stories \newline	
Unnecessary context switching \newline	
Inefficient tools and problematic APIs, libraries, and frameworks \newline	
Suffering from technical debt \newline	
Inefficient development flow \newline	
Poorly organized code \newline	
\\ \hline

Psychological Distress & The costs of burdening the team with unhelpful stress &  Emotional stress;  Low team morale; Rush mode
\\ \hline
Knowledge loss & The cost of re-acquiring information that the team once knew. & 
Knowledge loss from team churn \newline
Knowledge solos 
\\ \hline
Waiting                               & The cost of idle time, often hidden by multi-tasking. & Slow tests or unreliable tests \newline Unreliable acceptance environment \newline Missing  needed information, people, or equipment                                                                                                                                                                                                                                                                           \\ \hline
Ineffective communication             & The cost of incomplete, incorrect, misleading, inefficient, or absent communication.                         & 
Team size is too large \newline Asynchronous communication (distributed teams; distributed stakeholders; dependency on another team; opaque processes outside team) \newline Imbalance (dominating the conversation; not listening) \newline Inefficient meetings (lack of focus; skipping retros; not discussing blockers each day; meetings running over (e.g. long stand-ups)) \\ \hline                  
\end{tabular}
\end{table*}




We identified \numberOfWastes{} types of waste (Table \ref{Waste}). This section defines, elaborates, and provides examples of each type, including associated tensions where available.
\subsection{Waste: Building the wrong feature or product}
Building features (or worse, whole products) that no one needs, wants, or uses obviously wastes the time and efforts of everyone involved. We observed this waste affecting team morale, team code ownership \cite{SedanoTeamCodeOwnership}, and customer satisfaction. 

The product features for Project Ses were designed based on a given persona\textemdash i.e. a fictional, archetypal user \cite{Grudin2002personas}. However, consulting several real intended users revealed that the persona was deeply flawed as the users did not need the product. The intended users invalidated the persona. Building the intended product is risky and probably wasteful. 

\textbf{Tension: User needs} versus \textbf{business wants.}
Some projects exhibit a tension between user needs and business goals. Practitioners may struggle to produce something that simultaneously satisfies the users and the business.

On Project Quattuor, the client wanted to add a news feed to a mobile phone application that controlled a real world product in order to increase marketing awareness. However, user validation revealed that no users wanted this feature, and several reacted quite negatively. Despite numerous conversations, the marketing department insisted on adding the feature. 
\subsection{Waste: Mismanaging the backlog}
The product backlog can be mismanaged in several ways, leading to delays of key features or lower team productivity. 

On several projects, we observed engineers working on low-priority stories through \quotes{backlog inversion.} This occurs when the engineers working through the backlog get ahead of the product manager who is prioritizing the backlog. For instance, the product manager might prioritize the next ten stories in the backlog, but the engineers get to story 15 before the product manager gets back to prioritizing. This creates waste as engineers implement potentially outdated, low-value, or even counterproductive stories ahead of high-value stories.   

Mismanaging the backlog can also lead to duplicated work. We observed duplicate stories in the backlog, two engineers working on the same story because one had forgotten to change its status , and two engineers independently addressing the same pain point (e.g. making the build faster) by not communicating in the backlog what they were working on.

\textbf{Tension: Writing enough stories} versus \textbf{writing stories that will never be implemented.}
Pivotal product managers attempt to provide the team with a steady stream of ready, high-value work. This creates a tension between writing enough stories for the team to work on and \quotes{over-producing} stories that might never be implemented. Writing too few stories causes the team to idle while writing too many stories wastes the product manager's time. We observed teams running out of work on rare occasions; we did not observe product managers writing too many stories.  

\textbf{Tension: Finishing features} versus \textbf{working on too many features simultaneously.}
Product managers decompose a feature into a set of stories, and typically aim to create the minimal viable product as quickly as possible by sequencing the stories to finish just enough of each feature before starting another feature. 

On Project Quattuor's backend system, we observed one product manager starting too many tracks of work at once by prioritizing a breadth of features instead of finishing started features. Unfortunately, several tracks of work were not completed by the first release date. The work in progress was disabled with feature flags. Starting work, changing priorities, and halting work in flight, can result in waste.

We observed that teams usually prefer to maintain a shippable product while rapidly finishing the simplest possible version of each new feature.

\textbf{Tension: intransigence} versus \textbf{capricious adjustments.}
Responding to change quickly is a core tenet of agile development and often thought of as the opposite of refusing to change. However, responding to change is more like a middle ground between intransigence (unreasonably refusing to change) and thrashing (changing features too often, especially arbitrarily alternating between equally good alternatives). 

On Project Kvin, for example, the launch was delayed while the business fiddled with the sequence and number of steps in the user registration process. Project Duo was similarly delayed by a product manager repeatedly resequencing an order customization process. 
\subsection{Waste: Rework}
From a Waterfall perspective, one might classify any revision of existing code as \quotes{rework.} This problematically fails to distinguish between situations where things could have been done right based on the information available then from situations where new information reveals a better approach. 

Contrastingly, our participants classify revising work that should have been done correctly but was not as \textit{rework}, and improving existing work based on new information as \textit{new work}. \textit{Rework} wastes time and resources by definition. We observed numerous sources of \textit{rework} including stories with no clear definition of done, rejected stories (rejecting delivered work because the implementation did not satisfying the acceptance criteria), defects in the code, poor testing strategy, and ambiguous mock-ups.

On Project Ses, the engineers showed a finished story to the interaction designer for feedback. The interaction designer pointed out a missing interaction, which was neither in the story nor in the mock-up.

On Project Quattuor, the client delayed fixing of critical bugs until just before the release. Fixing one bug in the backend system had a cascading effect with the clients which expected the code to work a certain way. \textit{Rework} could have been avoided had the critical bug been fixed prior to the client code becoming dependent on it.

Also on Project Quattuor, the interaction designers created mockups optimized for English, not the target language. After implementing the application, the team realized that the target language text needed more space than the English translations, requiring rework for several design components. 

On Project Kvin, the interaction designer forgot to consider mobile phones when building the mock-ups. After building a few screens, the team realized that the website did not work well on mobile devices requiring \textit{rework}.
\subsection{Waste: Unnecessarily complex solutions}
\textit{Unnecessarily complex solutions} can be caused by feature complexity, technical complexity, or lack of reuse. Unnecessary feature complexity wastes users' time as they struggle to understand how to use the system and achieve their objectives, e.g. requiring the user to fill in form fields not related to the task at hand. Some features bring unnecessary technical complexity since a simpler interaction design would have solved the same problem. %\sout{We observed a simple solution was to increase conversations with product, design, and engineering. On one project, an engineer shortened the feedback loop by daily checking-in with the interaction designer.} Unnecessary technical complexity similarly wastes developers' time by making code more difficult to build and maintain. 

On Projects Tes, Ses, and Septem, complicated legacy components were refactored into simpler, easier to understand components. However, personal and organizational goals may misalign on this issue---one Pivotal engineer complained that a client engineer's attitude was, \participantQuote{the more complicated, the better, as that means my role is more important} \textemdash Participant 29.

Another way to increase system complexity is through a lack of reuse, i.e., building a new component instead of reusing an existing component. In code, lack of reuse can manifest as duplicated code and similar components that have similar functionality. In mockups, lack of reuse produces \quotes{snowflake designs}, designs which could take advantage of design reuse, e.g. two unique user interaction flows that could be unified into similar experiences or two visual components that solve the same concern.

On Project Duo, the interaction designer created a left-to-right navigational flow for configuring the product but designed a top-to-bottom navigational flow for the checkout page. Both sequences allowed the user to change a previous choice, jump to the correct page, and invalidate dependent information. In retrospect, using the same design treatment for both would have been faster. 

On Project Quattuor, multiple designers produced different design treatments for the same concept. The product shipped multiple versions of layouts, lists, alerts, and buttons, some with expensive interactions to delight users. 

On Project Kvin, the interaction designer created two sets of form inputs which necessitated multiple CSS styles for the HTML form input tags. Singular designs require engineering to build unique solutions with no possibility of reuse.   

\textbf{Tension: Big design up-front} versus \textbf{incremental design.}
Many projects exhibit a tension between up-front and incremental design. Rushing into implementation can produce ineffective emergent designs, leading to rework. However, big up-front design can produce incorrect or out-of-date assumptions and inability to cope with rapidly changing circumstances, also leading to expensive rework. The desire to avoid rework and differing development ideologies, therefore, motivate the tension and disagreement over big design up-front versus incremental design. 

The observed teams expected the product features to change even when the client had clearly defined the project. On all projects with interaction designers, after the interaction designer conducted user research and discovered new information about the user's needs, the feature set changed. No amount of up-front consideration appears sufficient to predict user feedback. The observed teams preferred delivering functionality incrementally and delay integrating with technologies until a feature requires it. For example, an engineer would only add asynchronous background jobs technology when working on the first story to require the needed technology, even if the team knew it would need it on day one of the project.

We observed teams using common architectural and design solutions from similar, previous projects without explicit architectural or design phases.

\subsection{Waste: Extraneous Cognitive Load}

Cognitive Load Theory \cite{Sweller1988} posits that our working memory is quite limited and overloading it inhibits learning and problem solving \cite{Artino2008}. Intrinsic cognitive load refers to the innate complexity of the task, while extraneous cognitive load refers to the cognitive load unnecessarily added by the task environment, or the way the task is presented \cite{Sweller2010}. Reducing the burden on working memory by removing extraneous cognitive load is therefore associated with more efficient learning among other positive outcomes \cite{VanSweller2005}.

Since many software development activities have high intrinsic cognitive load and developer's mental capacity is a limited resource, we view extraneous cognitive load as waste. While we cannot observe cognitive load directly, we did observe sources of extraneous cognitive load including overcomplicated stories, ineffective tooling, technical debt, and multitasking.

\textit{Overcomplicated stories}---user stories that are unnecessarily long, complex, unclear, or replete with pointers to other necessary information---are precisely the sort of task materials that Cognitive Load Theory predicts will overburden working memory.  On Project Quattuor, one story modifying the presentation of status resulted in the pair creating a spreadsheet listing out the complex behavior. The logic had become too complex to reason about in an individual's working memory. The team, concerned about code maintenance and readability, asked the product manager if simplifying the logic was possible.

\textit{Ineffective tooling} includes convoluted, nonfunctional, premature, complicated, unstable, outdated, unsupported, time-consuming, or inappropriate-for-the-task software libraries, as well as poorly designed development environments and deployment processes. Participant 13 said that one arcane technology \participantQuote{makes me angry enough that I want to hack into it, expose how useless and horrible it is and wipe this miserable product off the face of the earth!}.

\textit{Technical debt} refers to the risks of delaying needed technical work, by taking technical shortcuts, usually to meet a deadline \cite{McConnellTechnicalDebt}. Two key risks are the code being more difficult to understand and modify. We observed teams suffering from technical debt with long-running, existing code bases. On Project Tes for example, running the test suite produced 87,000 lines of output including deprecation warnings, exceptions, and test noise. Engineers ignored the overwhelming output which contained important information, On Project Ses, dead code littered the code base along with convoluted objects. Project Septem suffered from engineers introducing an idea in one part of the code base, but not applying the concept systematically. These examples illustrate how technical debt creates more things developers need to remember, in other words, how technical debt increases extraneous cognitive load. 

\textit{Multitasking}---performing two or more activities simultaneously or rapidly alternating between them---increases cognitive load as the multitasker attempts to holds two or more sets of information in working memory or needs to unnecessarily re-load  the information into working memory. While observed engineers prefer to finish one task before beginning the next task, they also try to convert excessive waiting (e.g. long builds, long tests, waiting for feedback) into productive time by multitasking (see \textit{waiting} waste description for more detail).

\subsection{Waste: Psychological Distress}
A distinct but related phenomena is psychological distress. “Stress is the nonspecific response of the body to any demand made upon it," \cite{Selye1976}. Stress may be beneficial (eustress) or harmful (distress). We see psychological distress as a kind of waste for the same reason as cognitive overload: developers are a limited resource, which distress consumes. Job-related psychological distress causes absenteeism, burnout, lower productivity and a variety of health problems \cite{Westman2001}.   
 
On Project Quattuor, for example, the team rushed to release a fixed feature set by a fixed date. Daily, the team decreased a countdown written on an office whiteboard to the release date. We observed low team morale, rush mode, lack of empathy, and waiting too long to resolve interpersonal issues leading to people working inefficiently. The team furthermore felt that over-emphasizing the deadline was increasing stress and leading to poor technical decisions, and eventually erased the countdown from the whiteboard. Participants felt that fixing both scope and schedule was antithetical to Pivotal's software process, where the client either chooses the release date and gets the features ready by then or chooses key features and ships the product when the features are ready. 

\subsection{Waste: Knowledge loss}
\textit{Knowledge loss} occurs when a team member with unique knowledge leaves a team or company---the latter being a more extreme form of knowledge loss.

In projects with knowledge silos, team churn leads to wasted effort as the team regains lost knowledge. For legacy systems, we observed teams sleuthing the code base, commit messages, and completed stories in the backlog to understand the code.  

On Project Octo, a complete team turnover required the new team to spend months understanding the system, during which the team's velocity was practically zero.

The observed teams reduced knowledge loss by actively removing knowledge silos and caretaking the code by adopting the principles, policies, and the practices of Sustainable Software Development \cite{SedanoSustainableSoftware}. Teams promoting knowledge sharing appear less susceptible to knowledge loss from team churn or team member rotation. 

\subsection{Waste: Waiting}
Having developers waiting around, working slowly or working on low-priority features because something is preventing them from proceeding on high-priority features wastes their time and delays their projects. For example, we observed developers waiting on (or looking for) product managers and designers to clarify a story's acceptance criteria. On Project Quattuor, product managers started multitasking while accepting stories because the acceptance environment was unreliable. We also saw team members waiting around because of missing video-conferencing equipment. 

Ohno described \textit{waiting} waste as hidden waste since people start working on the next job, instead of waiting \cite{OhnoToyotaProductionSystem}. To expose this waste, in Toyota Production System, when someone pulls the red cable, everyone stops, bringing attention to the waste. On Project Ses, it took 58 minutes to run the build locally and 17 minutes on the build machine due to parallelization on four machines. Team members would push code as a branch to the build machine instead of running tests locally. While the build machine ran the tests, the engineers would either wait or context switch onto different work. If the branch passed, some time later, they would merge their code into the team's code. If the branch failed, the engineers would decide either to finish the work that they were doing or switch back and fix the issue. Some engineers found the context switching exhausting. This \quotes{solution} for \textit{waiting} created \textit{extraneous cognitive load} waste.

\textbf{Tension: Wait, block, or guess.}
When needed information is missing, engineers appear to have three options: 1) wait for the information; 2) suspend (block) the story and work on something else; 3) act without the information. The best option depends on how far into the story the pair is, how long they have to wait, and their confidence in their guess.

\textbf{Tension: Waiting} versus \textbf{context switching.}
When possible, engineers would often use waiting time to attempt to remedy problems or reduce the duration of future waiting (e.g. shorten the build). When this was not possible, engineers often worked on something else instead of idling. Unfortunately, task switching decreases productivity and increases mistakes \cite{MonsellTaskSwitching}. For short waits, taking a break (e.g. playing table tennis) may be less wasteful than switching to another task. 

\subsection{Waste: Ineffective communication}
\textit{Ineffective communication} is incomplete, incorrect, misleading, inefficient, or absent communication. We observed that large team sizes, asynchronous communication, imbalance in communication, and inefficient meetings reduced team productivity.

We observed issues with asynchronous communication on Project Quattuor. The team was distributed between two offices separated by an hour commute. We observed the team using remote pairing and engineers commuting between the offices to mitigate the effects of a large distributed team, but communication issues continually arose in the retrospections.

On Project Ses, we observed that one person dominated meetings which prevented quieter personalities from sharing their perspective. 

On Project Quattuor, when the project started, the iOS team was not effectively reflecting on its process to make informed decisions. Adding weekly retros helped the team reflect and respond to problems. Over several weeks, the remaining teams added their own retros. 



%When teams are not co-located, additional time is spent on asynchronous communication. Instead of dialoguing about an issue, time is spent crafting a message, sending it, waiting for a response, and interpreting the response. When the message is misunderstood, resolving it takes longer than synchronous communication. When the team is distributed, it loses the benefits of osmotic communication.

%Increasing team size increases the number of communication paths. The number of paths is N x (N -1) / 2. 


% On Project Quattuor, \participantQuote{not enough listening in iOS technical meeting} appeared in two retros. 

% \textit{Ineffective communication} is a well studied topic. \cite{LencioniDeathByMeeting, CollaborationExplained, TabakaCollaborationExplained}.



\section{Comparing to Lean Software Development}
\label{LeanSoftwareDevelopmentComparison}

This section compares and contrasts our waste taxonomy (Section \ref{SEWaste}), with Lean Software Development's waste taxonomy \cite{PoppendieckConceptToCash} category by category. 

While several categories are analogous (Table \ref{LeanSoftwareDevelopmentComparisonTable}), we observed two types of waste not found in Lean: (\textit{unnecessarily complex solutions,} \textit{distress,} and \textit{ineffective communication}) (see Section \ref{SEWaste} for details). Meanwhile, we did not observe Lean's handoff waste type.

\begin{table}[t]
\renewcommand{\arraystretch}{1.5}
\centering
\caption{Comparison to Lean Software Development Waste}
\label{LeanSoftwareDevelopmentComparisonTable}
\begin{tabular}{|p{1.57in}|p{1.57in}|}
% \begin{tabular}{|l|l|} %dissertation size
\hline
Software Development Wastes           & Lean Software Development Wastes \\ \hline
Building the wrong feature or product & Extra Features                            \\ \hline
Mismanaging the backlog               & Partially Done Work                            \\ \hline
Unnecessarily complex solutions                & Not described                             \\ \hline
Rework                                & Defects                                   \\ \hline
Extraneous Cognitive Load                 & Task Switching  \\ \hline
Psychological Distress                             & Not described \\ \hline
Knowledge loss                 & Relearning                            \\ \hline
Waiting                               & Delays                                    \\ \hline
Ineffective communication             & Not described                             \\ \hline
Not observed                          & Handoffs                                  \\ \hline
\end{tabular}
\end{table}

% \begin{table}[t]
% \renewcommand{\arraystretch}{1.5}
% \centering
% \caption{Comparison to Lean Software Development Waste V2}
% \label{LeanSoftwareDevelopmentComparisonTable2}
% \begin{tabular}{|p{1.57in}|p{1.57in}|}
% % \begin{tabular}{|l|l|} %dissertation size
% \hline
% Software Development Wastes           & Lean Software Development Wastes \\ \hline
% Building the wrong feature or product & Extra Features                            \\ \hline
% Mismanaging the backlog               & Partially Done Work                            \\ \hline
% Unnecessarily complex solutions                & Not described                             \\ \hline
% Rework                                & Defects                                   \\ \hline
% Extraneous Cognitive Load                 & Not described  \\ \hline
% Psychological Distress                             & Not described \\ \hline
% Knowledge loss                 & Relearning                            \\ \hline
% Waiting \& Multitasking                              & Delays                                    \\  
%                   & Task Switching  \\ \hline
% Ineffective communication             & Not described                             \\ \hline
% Not observed                          & Handoffs                                  \\ \hline
% \end{tabular}
% \end{table}


\textbf{Handoffs}: We did not observe \textit{Handoff} waste---the loss of tacit knowledge when work is handed off to colleagues---as Pivotal follows an iterative software development process with cross functional teams. We did observe \textit{waiting} waste as engineers might contact people outside the team who had needed information. \textit{Handoffs} certainly contribute to the wastes of knowledge loss, ineffective communication, and waiting. However, our data does not support \textit{handoffs} as a \textit{type} of waste.

\textbf{Building the wrong feature or product} and \textbf{Extra features}: In our model, \textit{building the wrong feature or product} describes not addressing user or business needs. Pivotal's process relies on user validation to assess value in solving the user's needs and iterating from a minimal viable product. In Lean Software Development, \textit{extra features} describes adding in features that are not needed for the user's current needs. Both perspectives align on delaying features until necessary. \textit{Extra features} does not cover the waste from missing important business needs. \textit{Building the wrong feature or product} is a superset of \textit{extra features}.

\textbf{Mismanaging the backlog} and \textbf{Partially done work}: There is common ground between both taxonomies on reducing large batch sizes into smaller batches with an ideal of  \quotes{continuous flow,} where work is routinely moving through the system hence decreasing feature lead time. However, in Lean Software Development, \textit{partially done work} is work that is not tested, implemented, integrated, documented, or deployed. Any feature description that is not implemented, any code that is not integrated or merged, any code that is untested, any code that is not self-documenting or documented, and any code that is not deployed where the user can receive value is \textit{partially done work}.

\textit{Mismanaging the backlog} includes observed wastes beyond \textit{partially done work}, namely,  sequencing low priority work before high priority work, accidentally duplicating work, capricious thrashing, or delaying necessary bug fixes. Lean Software Development does not cover these wastes. 

The observed teams did not view materials flowing through the system as waste. Interaction designers need to produce just enough mockups, product managers need to write just enough stories, and developers need to write just enough code to make the story work. In any continuous flow system, there are unfinished materials at each step. 

While we did observe a product manager starting too many features at once (as described in the \textit{mismanaging the backlog} waste section), we mostly observed work flowing in a relatively orderly fashion with minimal delay. Large amounts of waiting designs or stories would have been classified as \textit{mismanaging the backlog} waste.

%Just-in-time production is antithetical to large batch sizes. In any continuous flow system, there is unfinished materials at each step. 

%Based on this understanding of continuous flow, we believe that \textit{\partially done work} is suggesting that large batch sizes need reducing.   

%The Toyota Production System creates buffers of parts at certain steps in the pipeline, like a bumper sits waiting to be used on the next car. Once that bumper is consumed, the process of creating another bumper just like it is started. Enough material is produced to keep continuous flow moving. Ohno was concerned about the stockpiles of inventory stored in warehouses.

%Since Pivotal engineers follow Test Driven Development / Behavior Driven Development, strive to get the tested code into the developer's master branch as quickly as possible, and release frequently, we did not observe the waste associated with partially done work. At any moment in time, an interaction designer is creating a future mock, a product manager is updating a future story, a developer is testing code that has not shipped. This describes the continuous flow of features to the customer. 

\textbf{Rework} and \textbf{Defects}: While both taxonomies agree on \textit{defects} being waste, our \textit{Rework} concept is a superset of \textit{defects}. \textit{Rework} includes mistakes made by the product managers (in writing acceptance criteria), the interaction designers (in creating mockups) and the developers (in writing tests and code). Poor testing strategies, and delaying testing can cause rework. Our definition of \textit{Rework} also distinguishes mistakes that could have been avoided from problems only obvious in hindsight. 

\textbf{Extraneous cognitive load} and \textbf{Task switching}: In our model, \textit{Extraneous cognitive load} includes the extra effort required by complex stories,  inefficient tools, suffering from technical debt, inefficient development processes, and context switching. In Lean Software Development, \textit{task switching} is the cost from trying to multitask. (Based on our analysis, task switching appears to be a cause, not a waste type.) The desire to have developers work on one thing at a time is common to both models. \textit{Extraneous cognitive load} subsumes \textit{task switching}. 

\textbf{Waiting} and \textbf{Delays}: In our model, \textit{waiting} includes delays from not having the needed information or resources to get one's work done as well as the cost of slow tests and unreliable tests. In Lean Software Development, \textit{delays} are \quotes{waiting for people to be available who are working in other areas} to provide needed information that is not available to the developers \cite{PoppendieckConceptToCash}. Both wastes describe the cost of missing needing information. \textit{Waiting} is a superset of \textit{delays}.

\textbf{Knowledge loss} and \textbf{Relearning}:
In our model, \textit{Knowledge loss} is the cost from members rolling off the team. In Lean Software Development, \textit{relearning} is \quotes{rediscovering something we once knew} \cite{PoppendieckConceptToCash}, which includes the cost of an individual not remembering a decision already made. The observed teams did not see forgetting as a waste.  Included in \textit{relearning} is failing to engage people in the development process. We did observe product managers having difficulty in involving some stakeholders, which we include in the \textit{building the wrong feature or product} waste. \textit{Knowledge loss} focuses \textit{relearning} to simply regaining departed knowledge. 
\section{Results Evaluation and Quality Criteria}
\label{ResultsEvaluation}
While other factors may affect software engineering waste, we focus only on those that we observed during the study. Grounded Theory studies can be evaluated using the following criteria \cite{Charmaz, StolGroundedTheory}:

\textbf{Credibility}: \quotes{Is there sufficient data to merit claims?}  This study relies on \durationOfResearchStudyPlural{} of participant-observation, \numberOfInterviews{} intensive open-ended interviews, and the agenda of one year's worth of retrospections. 

\textbf{Originality}: \quotes{Do the categories offer new insights?}  This is the first empirical study of waste in software development. It not only discovered new waste types but also supports and expands existing waste types that have not previously undergone rigorous empirical validation in a software engineering context. 

\textbf{Resonance}: \quotes{Does the theory make sense to participants?} We presented the results to the organization. Seven participants reviewed the results and this paper. They indicated that the waste taxonomy resonates with their experience: \quotes{These are pain points that we all felt. This covers what we learned on the projects}; \quotes{I have lived through these projects\ldots No other waste comes to mind.} This process produced no major changes, which is not surprising because participant observation mitigates resonance problems.

\textbf{Usefulness}: \quotes{Does the theory offer useful interpretations?} This study identifies software development wastes that are not identified in manufacturing and explains why certain behaviors, events, and actions can cause software engineering waste. It provides a rich waste taxonomy for identifying wastes in practice. %and for Value Stream Mapping. 

Regarding \textbf{external validity}, Grounded Theory is non-statistical, non-sampling research; therefore, the results cannot be statistically generalized to a population. Based on existing research and our experience, none of the identified wastes appear peculiar to Pivotal or Extreme Programming. However, Pivotal is a very effective organization which uses iterative development and has been concerned with eliminating waste for almost two decades. Organizations that are newer, less experienced, less concerned with waste, or use less iterative methods may experience additional waste types. For example \textit{waiting} waste manifests in organizations that hand feature documents between teams or use large batch sizes of features \cite{Ali2016, Khurum2014, Mujtaba2010}. Researchers and professionals should therefore take care to adapt our findings to their contexts case-by-case. 

Finally, the results might be influenced by \textbf{researcher bias} or \textbf{prior knowledge bias}. During participant-observation, the researcher may lose perspective and become biased by being a member of the team. That is, while a participant-observer gains perspective an outsider cannot, an outside observer might see something a participant observer will miss. Similarly, while prior knowledge helps the researcher interpret events and select lines of inquiry, prior knowledge may also blind the researcher to alternative explanations \cite{GlaserIssues}. We mitigated these risks by recording interviews and having the second and third authors review the coding process.
\section{Conclusion}
\label{Conclusion}
This paper presents the first evidence-based taxonomy of software engineering waste, identifies numerous causes of waste and explores fundamental tensions related to specific waste types. \numberOfWastes{} waste types are identified: building the wrong feature or product, mismanaging the backlog, rework, unnecessarily complex solutions, extraneous cognitive load, 
psychological distress, waiting, knowledge loss, and ineffective communication. Each waste is illustrated with examples taken from observed projects at Pivotal, showing how the waste materializes, and in some cases how it is removed or eliminated. We also compare our taxonomy to Lean Software Development's waste taxonomy.

Our taxonomy emerged from a Constructivist Grounded Theory study, including the collection and analysis of data from \durationOfResearchStudyPlural{} of participant-observation of  \numberOfObservedProjects{} software development projects, interviews of \numberOfInterviews{} software engineers, interaction designers, and product managers, as well as one year of retrospection topics. The analysis of the retrospection topics reveals that the observed Pivotal teams care deeply about finding and eliminating waste in their software development processes. 

While this research supports parts of the Lean Software Development waste taxonomy, it differs in three key ways: 1) it introduces two new waste categories: (\textit{unnecessarily complex solutions} and \textit{ineffective communication}); 2) it does not support LSD's \textit{Handoffs} waste category; 3) its waste categories are largely broader than LSD's. As such, our taxonomy is more expressive and more accurately describes the observed data. To be clear, the Lean Software Development's taxonomy of wastes was developed top-down, by mapping manufacturing wastes onto software development concepts. It was not empirically tested and therefore does not have the same epistemic status as our taxonomy, which was develop bottom-up from rigorous primary data collection and analysis. 

Several avenues for future research have potential. Studying different kinds of projects in different organizations may reveal new types of waste, or improve our understanding of existing categories. Furthermore, understanding wastes and their causes is just the first step in devising techniques for reducing waste and mitigating its effects. We would also like to investigate how teams use feedback loops to identify, manage, and reduce waste. 

%\sout{At the core of Lean Thinking is waste identification and elimination \cite{WomackLeanThinking} which is a key differentiator between Lean and Agile in software development \cite{Fitzgerald2015continuous}. (Agile does promote the use of feedback loops which we observed as a common solution for waste identification and removal at Pivotal.)

%Ohno and Shingo singularly focused on the efficiency of the production line through continuous waste removal. By excluding product development activities, the design of the vehicle, they overlooked the entire picture. This may explain why their waste taxonomy misses aspects fundamental to product design.

%In applying waste removal to software development, it behooves the software community first to start with a waste taxonomy grounded in data from software development, not borrow from a dissimilar domain. Since waste removal is core to the Lean movement, starting with an ill formulated waste taxonomy suggests a possible fundamental impact on Lean Software Development claims.

%This work suggests that evidence-based research yields insights grounded in data that have not emerged from applying manufacturing concepts to software development.}

%\sout{Based on the research from value stream mapping, the first step of waste reduction may be switching a software firm from a batch-and-pull process where a feature document is handed to a team to implement (e.g. a waterfall system) to a system where a small number of features are handed to a team (e.g. iterative software development.) Then introducing and reducing feedback loops to remove additional waste.}
\section*{Acknowledgement}
Thanks to Ben Christel for his assistance in sorting retro topics and helping with the initial analysis. Thank you to Rob Mee, David Goudreau, Ryan Richard, and Zach Larson for making this research possible.




\begin{table}[t]
\renewcommand{\arraystretch}{1.5}
\centering
\caption{Comparison to Lean Software Development Waste V2}
\label{LeanSoftwareDevelopmentComparisonTable2}
\begin{tabular}{|p{1.57in}|p{1.57in}|}
% \begin{tabular}{|l|l|} %dissertation size
\hline
Software Development Wastes           & Lean Software Development Wastes \\ \hline
Building the wrong feature or product & Extra Features                            \\ \hline
Mismanaging the backlog               & Partially Done Work                            \\ \hline
Unnecessarily complex solutions                & Not described                             \\ \hline
Rework                                & Defects                                   \\ \hline
Extraneous Cognitive Load                 & Not described  \\ \hline
Psychological Distress                             & Not described \\ \hline
Knowledge loss                 & Relearning                            \\ \hline
Waiting \& Multitasking                              & Delays                                    \\  
                  & Task Switching  \\ \hline
Ineffective communication             & Not described                             \\ \hline
Not observed                          & Handoffs                                  \\ \hline
\end{tabular}
\end{table}
